%% LyX 2.3.3 created this file.  For more info, see http://www.lyx.org/.
%% Do not edit unless you really know what you are doing.
\documentclass[11pt]{article}
\usepackage{mathpazo}
\usepackage[T1]{fontenc}
\usepackage[utf8]{inputenc}
\usepackage[active]{srcltx}
\usepackage{url}
\usepackage{graphicx}
\usepackage[unicode=true]
 {hyperref}

\makeatletter

%%%%%%%%%%%%%%%%%%%%%%%%%%%%%% LyX specific LaTeX commands.
\newcommand*\LyXZeroWidthSpace{\hspace{0pt}}

%%%%%%%%%%%%%%%%%%%%%%%%%%%%%% Textclass specific LaTeX commands.
\newenvironment{lyxlist}[1]
	{\begin{list}{}
		{\settowidth{\labelwidth}{#1}
		 \setlength{\leftmargin}{\labelwidth}
		 \addtolength{\leftmargin}{\labelsep}
		 \renewcommand{\makelabel}[1]{##1\hfil}}}
	{\end{list}}

%%%%%%%%%%%%%%%%%%%%%%%%%%%%%% User specified LaTeX commands.
%%%%%%%%%%%%%%%%%%%%%%%%%%%%%%%%%%%%%%%%%
% Academic Title Page
% LaTeX Template
% Version 2.0 (17/7/17)
%
% This template was downloaded from:
% http://www.LaTeXTemplates.com
%
% Original author:
% WikiBooks (LaTeX - Title Creation) with modifications by:
% Vel (vel@latextemplates.com)
%
% License:
% CC BY-NC-SA 3.0 (http://creativecommons.org/licenses/by-nc-sa/3.0/)
% 
% Instructions for using this template:
% This title page is capable of being compiled as is. This is not useful for 
% including it in another document. To do this, you have two options: 
%
% 1) Copy/paste everything between \begin{document} and \end{document} 
% starting at \begin{titlepage} and paste this into another LaTeX file where you 
% want your title page.
% OR
% 2) Remove everything outside the \begin{titlepage} and \end{titlepage}, rename
% this file and move it to the same directory as the LaTeX file you wish to add it to. 
% Then add \input{./<new filename>.tex} to your LaTeX file where you want your
% title page.
%
%%%%%%%%%%%%%%%%%%%%%%%%%%%%%%%%%%%%%%%%%

%----------------------------------------------------------------------------------------
%	PACKAGES AND OTHER DOCUMENT CONFIGURATIONS
%----------------------------------------------------------------------------------------



% Required for inputting international characters
% Output font encoding for international characters
\usepackage{graphics}
\usepackage{graphicx}
\usepackage{indentfirst}
\graphicspath{{images/} }
% Palatino font

\makeatother

\begin{document}
%----------------------------------------------------------------------------------------
%	TITLE PAGE
%----------------------------------------------------------------------------------------

\begin{titlepage} % Suppresses displaying the page number on the title page and the subsequent page counts as page 1
 
\global\long\def\HRule{\rule{\linewidth}{0.5mm}}%
 % Defines a new command for horizontal lines, change thickness here
 \centering % Centre everything on the page
 %------------------------------------------------
 %	Headings
 %------------------------------------------------
 \textsc{Portland State University}\\[1.5cm] % Main heading such as the name of your university/college
 \textsc{ECE 412}\\[0.5cm] % Major heading such as course name
 \textsc{Capstone Project Proposal 2020}\\[0.5cm] % Minor heading such as course title
 %------------------------------------------------
 %	Title
 %------------------------------------------------
 \HRule\\[0.4cm]

\textbf{\huge{}RISC-V Cleanflight Autopilot}\textbf{\\[0.4cm] % Title of your document
 \HRule\\[1.5cm]}

%------------------------------------------------
%	Author(s)
%------------------------------------------------
\noindent\begin{minipage}[c]{1\textwidth}%
 
\begin{center}
Bliss Brass {- brass@pdx.edu}\\
 % Your name
 
\par\end{center}%
\end{minipage}

\noindent\begin{minipage}[c]{1\textwidth}%
 
\begin{center}
Eric Schulte {- eschulte@pdx.edu}\\
 % Your name
 
\par\end{center}%
\end{minipage}

\noindent\begin{minipage}[c]{1\textwidth}%
 
\begin{center}
Nikolay Nikolov {- nnikolov@pdx.edu}\\
 % Your name
 
\par\end{center}%
\end{minipage}

\noindent\begin{minipage}[c]{1\textwidth}%
 
\begin{center}
Ruben Maldonado {- mruben@pdx.edu}\\
 % Your name
 
\par\end{center}%
\end{minipage}

\LyXZeroWidthSpace{}

\noindent\begin{minipage}[c]{1\textwidth}%

\begin{center}
\textit{\Large{}Advisor}{\Large{}:}\\
{\Large{} Roy Kravitz % Supervisor's name
 }{\Large\par}
\par\end{center}%
\end{minipage}\\
\LyXZeroWidthSpace{}

\noindent\begin{minipage}[c]{1\textwidth}%
 
\begin{center}
\textit{\Large{}Sponsor:}{\Large{}}\\
{\Large{} Galois % Supervisor's name
 }{\Large\par}
\par\end{center}%
\end{minipage}%------------------------------------------------
 %	Date
 %------------------------------------------------
 \vfill{}
\vfill{}
\vfill{}
{\large{}\today} % Date, change the \today to a set date if you want to be precise
 %------------------------------------------------
 %	Logo
 %------------------------------------------------
 \vfill{}
\vfill{}
\vfill{}
\end{titlepage} \pagestyle{plain} \tableofcontents{}\newpage{}

\section{Executive Summary}

Our objective is to port Cleanflight, a mini open-source OS for drones to a board running on the RISC-V architecture. We are going to achieve our aim by turning SiFive’s HiFive1 Revision B RISC-V development board into the flight controller for a standard 4-propeller racing drone. 

The successful completion of the project will act as a proof-of-concept for the newer RISC-V architecture, as compared to older RISC machines like ARM.

Our project sponsor, Galois Inc., has asked us to deliver a flying racing drone with the HiFive1 Rev. B board running Cleanflight by the end of the spring term in June 2020. Our sponsor, Galois, is purchasing and providing a drone kit for us to replace the native flight controller with the HiFive1 Rev. B.

\section{Project Design Specification}

\subsection{Concept of operations}

The primary purpose of this project is to demonstrate RISC-V’s capabilities in a fun and exciting way. The primary users will be Galois and possibly other drone hobbyists looking to adapt future RISC-V boards for use on racing drones. It may also be used by others as a reference or base-point to launch their own RISC-V based projects.

\subsection{Stakeholders}

The main project stakeholder is our project sponsor, Galois Inc. We will be providing them with a functioning drone running on RISC-V architecture. The other stakeholder (though they may not know it yet) is the Cleanflight development team. We will share our RISC-V port via Github pull request at the end of the project.

\subsection{Background}

\subsubsection{Cleanflight}
Cleanflight is an open-source flight control software for drones and fixed-wing aircraft. Written in the C language, Cleanflight currently supports over 100 flight control board models. Users compile and build model-specific versions of Cleanflight through a Google App or using a command line interface. Cleanflight currently only supports flight control boards with STM32 RISC microcontrollers.

\subsubsection{RISC-V Architecture}
RISC-V is an open-source Instruction Set Architecture (ISA) developed at UC Berkeley and introduced in 2010. RISC-V is a modular ISA centered around the philosophy of ‘use only what you need,’ meaning the ISA aims to minimize wasteful use of resources. RISC-V includes a static core, RV32I, with the option of adding one or more standard extensions that hardware can consist of or not based on the needs of an application. The modular concept enables very small and low energy implementations of RISC-V, which can be critical for embedded applications.

In recent years RISC-V has even drawn the interest of companies such as Apple, Facebook, Google, and Samsung. Interested in the modular philosophy of the ISA, these companies have begun investing resources into creating their own RISC-V based silicon.
The industry currently requires more practical applications to demonstrate the capabilities of RISC-V. We aim to create a racing drone to serve as one of these demonstration platforms.

\subsection{Market Analysis}

All drones, racing or otherwise, are controlled using a flight control board. The flight control board is a PCB that acts as the ‘brains’ of the drone; receiving the user input from an RC transmitter on the ground, processing the data, and outputting control instructions to the motors. All flight control boards must include at least two components: a processing unit, and an inertial measurement unit (IMU). Additional features may consist of components such as cameras, compasses, barometers, and telemetry devices.

The popularity of drones has exploded over the last ten years. As a result, the flight controller market has grown to include a diverse set of products from hundreds of companies. It’s now possible to find a quality flight controller for between \$25.00 - \$50.00.

By porting Cleanflight to the HiFive1 Rev. B development board, our team is essentially creating a custom flight control board. Our board won’t be the first RISC-V flight controller to market, but, as far as we can tell, it will be the first RISC-V board to support Cleanflight.

We’ll be modeling our board’s functionality around that of the Seriously Pro Racing F3 Flight Controller (SPRacingF3). The SPRacingF3 includes an accelerometer and gyroscope for determining orientation and speed, a compass for determining direction, and a barometer for determining altitude. We plan on completely implementing the accelerometer and gyroscope, while the compass and barometer remain stretch goals.

\subsection{Requirements}

Most broadly, the drone must fly with the HiFive1 Rev. B running a ported version of Cleanflight. More specifically, the drone must respond accurately and responsively to user input controlling speed, direction, and altitude.

Since Cleanflight is an open-source software project, our team must also document the porting process thoroughly enough so that someone else could recreate the project. We will share the final documentation with the Cleanflight development team via a Github merge request at the end of the project. As such, a Cleanflight user should be able to use the CleanFlight tool to flash their own RISC-V based flight control board.

\subsection{Specifications}

Our team must use Cleanflight as the flight control software, and use the HiFive1 Rev. B board as the flight control board. These are the only specifications provided by our project sponsor.


\subsection{Deliverables}

\begin{itemize}
    \item Project Proposal
    \begin{itemize}
        \item Product Design Specification
         \item Project Management Plan
    \end{itemize}
    \item Weekly progress reports
    \item Test plan
    \item Test results report
    \item Final report
    \item ECE Capstone poster session poster
    \item Github merge request (to incorporate the RISC-V port into Cleanflight’s mainstream repository)

\end{itemize}

\subsection{Initial Designs}
\subsubsection{Drone Kit Block Diagram}

Below is the high-level block diagram for the assembled drone kit. The power distribution board (PDB) takes 11.1V power from the LIPO battery and converts it to 5V, powering the receiver, flight control board and the electronic speed controllers (ESCs). The receiver detects signals from the transmitter on the ground and relays data to the flight control board via a UART interface. The flight control board also receives input from the MPU 9250 accelerometer/gyroscope, via a SPI interface. The flight control board then outputs control data to the four ESCs in the form of PWM signals. Level shifters are required to step down input signal voltage from the receiver, and to step up output signal voltage to the ESCs. Lastly, the four propeller motors are connected to the four ESCs.

\begin{figure}[ht]
  \caption{Drone kit block diagram.}
  \centering
    \includegraphics[width=150mm,scale=1]{Block diagram V2.jpg}
\end{figure}

\subsubsection{Cleanflight’s Makefile}

Cleanflight’s Makefile provides instructions to the make utility on how to compile and build the final Cleanflight executable file. This executable file then gets flashed to the target flight control board.

To create an executable for the RISC-V environment, our team must integrate RISC-V support into the Makefile. The first step in that process is exploring how the Makefile tells the compiler to build an executable. Below is a high-level block diagram of the Cleanflight Makefile’s functionality:

\begin{figure}[ht]
  \caption{Makefile flowchart.}
  \centering
    \includegraphics[width=50mm,scale=1]{make.png}
\end{figure}


\subsubsection{The Minimum Port}

Before porting any specific Cleanflight drivers, we want to complete a ‘dry run’ of the porting process. This will involve editing Cleanflight’s makefile to include RISC-V development tools and flags, and getting the makefile to produce an executable for the HiFive1 Rev. B. The idea is to learn how to build RISC-V executables using a version of Cleanflight’s makefile, prior to attempting to port any specific software. We refer to this initial dry run as the Minimum Port, and its steps are listed below:

\begin{figure}[ht]
  \caption{Minimum port flowchart.}
  \centering
    \includegraphics[width=40mm,scale=0.5]{min_port.png}
\end{figure}

\newpage

\begin{enumerate}
    \item Start with the Cleanflight makefile.
    
    \item Download and include a path to the prebuilt RISC-V toolchain which includes GCC, G++, GDB, etc.
    
    \item The toolchain comes with Freedom Studio, or it can be found on SiFive’s website. The path can then be hardcoded in the makefile, or the tool.mk file can be updated to find the toolchain.
    
    \item Add paths to .s and .S files. These files can be found after compiling a sample app with HiFive1 Rev. B listed as the target.
    
    \item Add Cflags, Ldflags, etc. These files can be found after compiling a sample app with HiFive1 Rev. B listed as the target.
    
    \item Add metal libs/includes files. These files can be found after compiling a sample app with HiFive1 Rev. B listed as the target.
    \item Build a simple example program (i.e. flashing LEDs)
    \item Exit makefile

\end{enumerate}

\subsection{Verification Plan}

After completing the minimum port, we plan on splitting the porting work into modules. 

The first module to be ported is the receiver firmware. Once that is complete, we will unit-test the firmware with the FS-iA6B Receiver to ensure standalone functionality. After the port of the receiver module, the team will split into two-person subteams, each focusing on one module. 

One team will be responsible for porting the accelerometer firmware, and the other responsible for porting the ESC/motor output firmware. Similar to the receiver module testing, each team will unit-test their module to check for standalone functionality.

Once we confirm each module is working on its own, we will remove the stock flight control board from the drone and install the HiFive1 Rev. B as the flight controller. The final step will be performing integration testing to ensure all modules function correctly together as a single package. 

These are just preliminary verification plans, with more specifics to come. But as of right now, this is the general outline of our verification plan.

\subsection{Risks}

Our team’s biggest risk is our mutual lack of porting experience. We all have experience with software development, but no one has actually ported software before. This unfamiliarity may introduce confusion and uncertainty throughout the project. Therefore, it’s important we develop an overall strategy before diving into any specific code (i.e. first creating a “minimum port” discussed above).

Another risk comes from the capabilities of the HiFive1 Rev. B itself. For example: are there enough GPIO pins and SPI ports to support Cleanflight? Will the program memory of the HiFive1 Rev. B be large enough for the final executable? These are the types of issues that could arise later in the porting process.

Lastly, is the issue of compatibility between the stock drone kit and the HiFive1 Rev. B. Out of the box, the kit is designed to support an ARM-based flight controller. While the HiFive1 Rev. B should provide all the features of an ARM microcontroller, it’s possible there may be a few compatibility issues.


\newpage

\section{Project Management Plan}

\subsection{Timeline}

\begin{figure}[h!]
  \caption{Project schedule for winter and spring terms.}
  \centering
    \includegraphics[width=170mm,scale=1]{Project Schedule V2.jpg}
\end{figure}

\newpage 

\subsection{Milestones}

\begin{enumerate}
    \item Complete research phase
    \item Complete Minimum Port (see PDS, Initial Designs section for more details)
    \item Project Proposal approved by sponsor
    \begin{enumerate}
        \item PDS
        \item Project Management Plan
    \end{enumerate}
    
    \item Assemble drone kit and install stock flight control board
    \begin{enumerate}
        \item Flash Cleanflight to stock board
        \item Test flight capabilities
    \end{enumerate} 
    
    \item Receiver module tested and ported to HiFive1 Rev. B
    \item Accelerometer/Gyroscope module tested and ported to HiFive1 Rev. B
    \item ESC/motor output module tested and ported to HiFive1 Rev. B
    \item Install HiFive1 Rev. B as flight control board
    \item Complete final integration testing
    \item Create final documentation:
    \begin{enumerate}
        \item RISC-V Github pull request
        \item Final report
        \item Presentation poster
    \end{enumerate}
\end{enumerate}

\newpage

\subsection{Project Budget and Resources}
\subsubsection{BOM}

\resizebox{\textwidth}{!}
{
\begin{tabular}{|c|c|c|c|c|} \hline  Item & Description & Qty. & Price & Total\tabularnewline \hline  \hline  LHI 280 Race Quad ARF  & Drone kit (including SPRacingF3 FC) & 1 & \$105.99 & \$105.99\tabularnewline \hline  Flysky FS-i6X 10CH 2.4GHz AFHDS RC Transmitter w/ FS-iA6B Receiver & RC transmitter and receiver & 1 & \$52.99 & \$52.99\tabularnewline \hline  Ovonic 11.1V 2200mAh Lipo Battery & Battery & 1 & \$17.99 & \$17.99\tabularnewline \hline  Sparkfun MPU-9250 Breakout board & Accelerometer/gyroscope & 1 & \$14.95 & \$14.95\tabularnewline \hline  HiFive1 Rev. B Development Board & RISC-V development board & 2 & \$59.00 & \$118 .00\tabularnewline \hline  Adafruit Proto-screwshield & RISC-V Protoboard & 1 & \$14.95 & \$14.95\tabularnewline \hline  Total &  &  &  & \$324.87 \tabularnewline \hline  \end{tabular}

}
\newline

Our total project budget comes out to roughly \$325.00. This budget includes: 
\begin{enumerate}
    \item (1) Racing drone kit
    \item (1) RC transmitter and receiver
    \item (1) LIPO battery
    \item (1) MPU-9250 accelerometer breakout board
    \item (2) HiFive1 Rev. B Development Boards
    \item (1) Protoboard
\end{enumerate}

We want to first assemble the stock drone kit (including the stock flight control board), and flash a standard build of Cleanflight to get a better idea of the build process for a specific target.

This will hopefully give us a better understanding of how to do the same process with the HiFive1 Rev. B, the only difference being the ported software.

We will use the Capstone Lab and the EPL during the hardware assembly and testing phases.

\newpage

\subsection{Team members and relevant skills}

\begin{lyxlist}{}

\item Bliss Brass, CE :
        \begin{itemize}
             \item C,C++ scripting,LaTeX,Markdown
             \item Firmware and Linux device drivers
             \item Microprocessor Architecture
        \end{itemize}
        
 
    
\item Ruben Maldonado, CE :
        \begin{itemize}
            \item C,C++,Python, Shell scripting
            \item Firmware and Linux device drivers
            \item Microprocessor Architecture
        \end{itemize}

\item Nikolay Nikolov, CE : 
    \begin{itemize}
        \item C,C++,Python, Shell scripting,LaTeX,Markdown
        \item Firmware and Linux device drivers
        \item Microprocessor Architecture
    \end{itemize}
 
\item Eric Schulte, EE : 
    \begin{itemize}
        \item Microprocessor system design (ECE 371)
        \item Hardware prototyping and testing
        \item Some object-oriented coding (C/C++)
    \end{itemize}
    
\end{lyxlist}

\newpage

\subsection{Development Process}

\emph{Note: These plans are preliminary and subject to change, but this is the overall idea of how we expect the project to proceed.}

Ruben has volunteered to act as liaison between the team and our industry sponsor and faculty advisor. We are using Slack for daily communication, and Github’s Project feature for progress/task tracking.
 
The plan is to have everybody working on tasks in series for most of the project. Since we’re all new to software porting we think this is the best strategy to minimize setbacks. The first thing we’re going to ‘build’ is what we call the ‘minimum port’. 
 
The minimum port entails creating a basic makefile, based on Cleanflight’s makefile, to build an executable of a simple demo program (i.e. flash some LEDs), and then flash it to the HiFive1 Rev. B using SiFive’s Freedom Studio. This simple makefile will serve as our team’s entry-point into the Cleanflight code. Once we are able to compile and build a simple demo program we can add more complexity and start building advanced executables like ported hardware drivers.
 
With our custom makefile complete, our team will begin porting the necessary software for the FS-iA6B Receiver. This device relays signals from the transmitter on the ground to the drone’s flight control board. Since the receiver firmware will be our team’s first port, we’ve made this phase the longest of the project. We anticipate this phase will be quite challenging; as such, our team plans to work in series.
	
Once the receiver port is complete, we’ll begin focusing on porting firmware for the accelerometer and the electric speed controllers (ESC). ESCs act as an intermediary device between the flight control board and motors controlling the propellers. Hopefully by this point the porting process will be better understood, which is why we plan on splitting into teams of two to maximize efficiency. One team will work on porting accelerometer firmware, while the other ports firmware for the ESCs.
	
We plan to incorporate three phases of testing throughout the project: One phase of unit-testing for the receiver module, another for the accelerometer and ESC modules, and a third phase for final integration testing. As part of the integration testing, we will swap the kit’s stock flight control board (SPRacingF3) with the HiFive1 Rev. B running the ported Cleanflight firmware.
 
We plan to integrate modules cumulatively, meaning that as soon as we get one up and running we will immediately try to integrate it with the others. We hope this iterative process will result in less debugging effort needed during the final month of the project.
	
After final integration testing is complete, we will focus on compiling our project’s final documentation. This will include: assembling a Github merge request to submit our RISC-V port to the mainstream of Cleanflight, writing the final report, and creating the presentation poster.
   
\emph{Note: We should be completing necessary documentation (i.e., wikis, manuals) for the Github merge request along the way, not all at once at the end.}

\end{document}
